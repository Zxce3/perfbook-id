% howto/howto.tex
% mainfile: ../perfbook.tex
% SPDX-License-Identifier: CC-BY-SA-3.0

\QuickQuizChapter{chp:How To Use This Book}{Cara Menggunakan Buku Ini}{qqzhowto}
%
\Epigraph{Jika kamu menyadari bahwa hidup itu sulit, maka segalanya akan
	  jadi lebih mudah.}{Louis D. Brandeis}

Tujuan dari buku ini adalah untuk membantu Kamu mengembangkan
program untuk sistem paralel dengan shared-memori tanpa mengorbankan
kewarasan Kamu.\footnote{
	Atau, mungkin lebih tepatnya, tanpa mengorbankan kewarasan Kamu
	dengan lebih banyak risiko daripada yang dihadapi oleh programmer
	non-paralel.
	Yang, sekarang saya berpikir, mungkin tidak berarti banyak.}
Namun, Kamu harus menganggap informasi dalam buku ini sebagai
dasar untuk dibangun, bukan sebagai sebuah katedral yang selesai.
Misi Kamu, jika Kamu memilih untuk menerima, adalah untuk membantu
proses pembangunan kedepan dalam bidang pemrograman paralel---proses
yang akan pada akhirnya membuat buku ini usang.

Pemrograman paralel di abad ke 21 ini sudah tidak lagi berfokus pada
ilmu, penelitian, dan proyek tantangan besar.
Dan ini semua baik, karena berarti bahwa pemrograman paralel menjadi
disiplin ilmu pengetahuan dan teknologi.
Oleh karena itu, seperti yang pantas bagi disiplin ilmu pengetahuan dan teknologi,
buku ini memeriksa tugas-tugas pemrograman paralel khusus dan menggambarkan
bagaimana menangani mereka.
Dalam beberapa kasus yang sangat umum, tugas-tugas ini dapat diotomatisasi.

Buku ini ditulis dengan harapan bahwa penjelasan disiplin ilmu
yang mendasari proyek-proyek pemrograman paralel yang sukses akan membebaskan
generasi baru hacker paralel dari kebutuhan untuk secara perlahan dan
berulang-ulang menginventarisasi kembali roda-roda lama, memungkinkan mereka
untuk fokus pada bidang baru.
Namun, apa yang kamu dapatkan dari buku ini akan ditentukan oleh apa yang
kamu masukkan ke dalamnya.
Ini diharapkan bahwa hanya membaca buku ini akan membantu,
dan bekerja pada kuis cepat akan lebih membantu.
Namun, hasil terbaik datang dari menerapkan teknik-teknik yang diajarkan
dalam buku ini ke masalah-masalah nyata.
Seperti biasanya, latihan membuat sempurna.

Tetapi, bagaimana pun kamu menggunakannya, kami berharap dengan pemrograman
paralel kamu akan mendapatkan hiburan, kegembiraan, dan tantangan yang sama
yang telah kami dapatkan!

\section{Roadmap}
\label{sec:howto:Roadmap}
%
\epigraph{Kucing:
		Kamu mau kemana? \\
	  Alice:
		Apa arah yang harus aku ambil? \\
	  Kucing:
		Tergantung dari mana kamu mau pergi. \\
	  Alice:
		Aku tidak tahu. \\
	  Kucing:
		Maka tidak ada perbedaan mana arah yang harus kamu ambil.}
	{Lewis Carroll, \emph{Alice in Wonderland}}	

Buku ini adalah panduan yang secara luas dan banyak digunakan untuk
teknik-teknik desain, bukan kumpulan algoritma yang optimal dengan
area aplikasi yang sangat kecil.
Kamus sekarang sedang membaca \cref{chp:How To Use This Book}, tapi
kamu tahu itu sudah memberikan gambaran tingkat tinggi tentang \Cref{chp:Introduction}
pemrograman paralel.

\Cref{chp:Hardware and its Habits} pengenalan tentang perangkat keras
yang berbagi memori paralel.
Setelah semuanya, sulit untuk menulis kode paralel yang baik kecuali
kamu memahami dasar dari perangkat keras.
Karena perangkat keras terus berkembang, bab ini akan pasti
ketinggalan zaman.
Kami akan tetap berusaha untuk tetap terkini.
\Cref{chp:Tools of the Trade} kemudian memberikan ikhtisar primitif-primitif
yang singkat tentang pemrograman paralel secara umum.

\Cref{chp:Counting} mengambil pandangan mendalam pada masalah
paralel yang paling sederhana, yaitu perhitungan.
Karena hampir semua orang memiliki pemahaman yang baik tentang
perhitungan, bab ini dapat menyelidiki banyak masalah
pemrograman paralel yang penting tanpa gangguan dari masalah
komputer ilmiah yang lebih umum.
Impresiku adalah bahwa bab ini telah digunakan dalam
mata kuliah pemrograman paralel.

\Cref{cha:Partitioning and Synchronization Design}
memperkenalkan beberapa metode desain untuk mengatasi masalah
yang diidentifikasi dalam \cref{chp:Counting}.
Ternyata penting untuk mengatasi paralelisme pada tingkat
desain jika memungkinkan:
Untuk mengutip \pplsur{Edsger W.}{Dijkstra}~\cite{Dijkstra:1968:LEG:362929.362947},
``retrofitted parallelism considered grossly
suboptimal''~\cite{PaulEMcKenney2012HOTPARsuboptimal}.

Tiga bab berikutnya memeriksa tiga bagian penting untuk
sinkronisasi.
\Cref{chp:Locking} membahas locking, yang bukan hanya
workhorse dari pemrograman paralel untuk kualitas produksi, tetapi juga
sering dianggap sebagai hal yang buruk di pemrograman pararel.
\Cref{chp:Data Ownership} memberikan ikhtisar singkat tentang
pemilikan data, bagian yang sering diabaikan tetapi
mempunyai kekuatan yang luar biasa dan sangat luas.
Akhirnya, \cref{chp:Deferred Processing} memperkenalkan beberapa
mekanisme deferred-processing, termasuk reference counting,
hazard pointers, sequence locking, dan RCU\@.

\Cref{chp:Data Structures} menerapkan pelajaran dari bab-bab
sebelumnya ke tabel hash, yang sering digunakan karena
partisi yang sangat baik, yang (biasanya) mengarah pada kinerja
dan skala yang luar biasa.

Seperti yang telah dipelajari banyak orang untuk kesedihannya,
pemrograman paralel tanpa validasi adalah jalan yang pasti menuju
kegagalan yang abjek.
\Cref{chp:Validation} membahas berbagai bentuk pengujian.
Tentu saja tidak mungkin menguji keandalan ke programmu
setelah fakta, jadi \cref{chp:Formal Verification}
mengikuti dengan ikhtisar singkat dari beberapa bagian
praktis untuk verifikasi formal.

\Cref{chp:Putting It All Together}
berisi serangkaian masalah pemrograman paralel berukuran sedang.
Kesulitan masalah-masalah ini bervariasi, tetapi seharusnya
cocok bagi seseorang yang telah menguasai materi dalam bab-bab
sebelumnya.

\Cref{sec:advsync:Advanced Synchronization}
melihat metode sinkronisasi lanjutan, termasuk sinkronisasi
non-blocking dan komputasi real-time paralel,
sedangkan \cref{chp:Advanced Synchronization: Memory Ordering}
mengulas topik lanjutan tentang pengurutan memori.
\Cref{chp:Ease of Use} mengikuti dengan beberapa saran
kemudahan penggunaan.
\Cref{chp:Conflicting Visions of the Future}
melihat beberapa arah masa depan, yang mungkin termasuk
desain sistem shared-memory paralel, transaksi perangkat lunak
dan perangkat keras, dan pemrograman fungsional untuk
paralelisme.
Akhirnya, \cref{chp:Looking Forward and Back} meninjau
materi dalam buku ini dan asal-usulnya.

Bab ini diikuti oleh beberapa lampiran.
Yang paling populer dari mereka sepertinya adalah
\cref{chp:app:whymb:Why Memory Barriers?},
yang mendalami memory-ordering.
\Cref{chp:app:Answers to Quick Quizzes}
mengandung jawaban dari kuis cepat yang sangat terkenal, yang
dibahas dalam bagian berikutnya.

\section{Quick Quizzes}
\label{sec:howto:Quick Quizzes}
%
\epigraph{Lakukan sesuatu yang sulit, jika tidak Kamu tidak akan
	berkembang.}
	 {Singkat dari Ronald E.~Osburn}

``Quick quizzes'' muncul di seluruh buku ini, dan jawabannya
dapat ditemukan di
\cref{chp:app:Answers to Quick Quizzes} dimulai dari
\cpageref{chp:app:Answers to Quick Quizzes}.
Beberapa dari mereka didasarkan pada materi di mana kuis cepat
itu muncul, tetapi beberapa lainnya memerlukan Kamu untuk berpikir
di luar bagian itu, dan, dalam beberapa kasus, di luar cakupan
pengetahuan saat ini.
Seperti kebanyakan upaya, apa yang Kamu
dapatkan dari buku ini tergantung pada apa yang Kamu siapkan untuk
memasukkannya.

Oleh karena itu, pembaca yang membuat usaha yang sungguh-sungguh
untuk menyelesaikan kuis sebelum melihat jawabannya
mendapatkan upaya mereka dibayar dengan
pemahaman yang meningkat tentang pemrograman paralel.


\QuickQuizSeries{%
\QuickQuizB{
	Di mana Kamu dapat menemukan jawaban dari kuiz cepat?
}\QuickQuizAnswerB{
	Di \cref{chp:app:Answers to Quick Quizzes} mulai
	\cpageref{chp:app:Answers to Quick Quizzes}.
	Hai, saya pikir saya harus memberi Kamu yang mudah!
}\QuickQuizEndB
%
\QuickQuizM{
	Beberapa pertanyaan kuis cepat tampaknya berasal dari sudut pandang
	pembaca daripada penulis.
	Apakah itu benar-benar niatnya?
}\QuickQuizAnswerM{
	Ya Benar!
	Banyak pertanyaan yang mungkin dimiliki oleh Paul E. ~ McKenney
	Dia ditanya apakah dia adalah seorang mahasiswa pemula di kelas yang mencakup materi ini.
	Perlu dicatat bahwa Paulus diajari sebagian besar materi ini oleh
	Perangkat keras dan perangkat lunak paralel, bukan oleh profesor.
	Dalam pengalaman Paul, profesor jauh lebih mungkin untuk menyediakan
	Jawaban atas pertanyaan verbal daripada sistem paralel, baru -baru ini
	Kemajuan dalam asisten yang diaktifkan suara.
	Tentu saja, kita bisa berdebat panjang tentang profesor mana
	atau sistem paralel memberikan jawaban yang paling berguna untuk jenis
	pertanyaan ini,
	tetapi untuk saat ini, mari kita setujui hanya untuk kegunaannya
	Jawaban sangat bervariasi di seluruh populasi kedua profesor
	dan sistem paralel.

	Kuis lain sangat mirip dengan pertanyaan aktual yang telah terjadi
	ditanya selama presentasi konferensi dan kuliah yang mencakup
	materi dalam buku ini.
	Beberapa orang lain berasal dari sudut pandang penulis.
}\QuickQuizEndM
%
\QuickQuizE{
	Kuis cepat ini bukan secangkir teh saya.
	Apa yang bisa saya lakukan?
}\QuickQuizAnswerE{
Berikut adalah beberapa strategi yang mungkin Kamu pertimbangkan:

\begin{enumerate}
\item	Abaikan saja kuis cepat dan baca isi buku ini.
	Kamu mungkin melewatkan materi yang menarik
	beberapa kuis cepat, tetapi sisa buku ini
	memiliki banyak bahan bagus juga.
	Ini adalah pendekatan yang sangat masuk akal jika utama Kamu
	Tujuannya adalah untuk mendapatkan pemahaman umum tentang materi
	atau jika Kamu membaca sekilas buku ini untuk menemukan
	solusi untuk masalah tertentu.
\item	Lihatlah jawaban segera setelah Kamu menyelesaikan kuis cepat.
	Ini adalah pendekatan yang masuk akal jika kuis cepat
	dalam suatu bab memiliki kunci untuk masalah yang Kamu
	coba selesaikan.
	Ini juga masuk akal jika Kamu ingin pemahaman yang sedikit lebih dalam
	materi, tetapi ketika Kamu tidak berharap untuk diuji untuk
	menghasilkan solusi yang sama diberikan hanya lembaran kosong.
\item   Jika kamu menemukan kuis cepat yang mengganggu tetapi tidak mungkin
	diabaikan, kamu bisa selalu mengkloning \LaTeX{} sumber untuk
	buku ini dari arsip git.
	Kamu bisa kemudian menjalankan perintah \co{make nq}, yang akan
	menghasilkan \co{perfbook-nq.pdf}.
	PDF ini berisi kotak-kotak yang tidak mengganggu tag di mana kuis cepat
	sebaliknya, dan mengumpulkan setiap bab kuis cepat
	pada akhir bab itu dalam gaya buku teks klasik.
\item	Latihlah untuk menyukai (atau setidaknya menoleransi) kuis cepat.
	Pengalaman menunjukkan bahwa menguji diri sendiri secara berkala
	saat membaca sangat meningkatkan pemahaman dan kedalaman
	pemahaman.
\end{enumerate}

Catat bahwa kuis cepat dihubungkan ke jawaban dan sebaliknya.
Klik judul ``Quick Quiz'' atau kotak hitam kecil
untuk pindah ke awal jawaban.
Dari jawaban, klik judul atau kotak hitam kecil untuk
pindah ke awal kuis, atau, alternatif, klik kotak putih kecil
di akhir jawaban untuk pindah ke akhir
kuis yang sesuai.
}\QuickQuizEndE
}

Singkatnya, jika Kamu membutuhkan pemahaman yang mendalam
dari materi, Kamu harus menginvestasikan waktu
untuk menjawab kuis cepat.
Jangan salah paham, membaca secara pasif materi dapat sangat
berharga, tetapi mendapatkan kemampuan pemecahan masalah yang lengkap benar-benar
membutuhkan Kamu untuk berlatih memecahkan masalah.
Hal yang sama berlaku untuk mendapatkan kemampuan produksi kode yang lengkap
benar-benar membutuhkan Kamu untuk berlatih menghasilkan kode.


\QuickQuiz{
	Jika hanya membaca buku ini secara pasif tidak membuat saya
	mendapatkan kemampuan pemecahan masalah dan produksi kode yang lengkap,
	apakah ada artinya???
}\QuickQuizAnswer{
	Bagi mereka yang lebih suka analogi, pemrograman perangkat lunak
	bersamaan mirip dengan memainkan musik dalam hal banyak
	kegunaan yang baik untuk banyak tingkat keterampilan dan keterampilan.
	Tidak semua orang perlu menghabiskan seluruh hidupnya menjadi
	seorang pianis konser.
	Dalam hal ini, untuk setiap pianis virtuoso seperti itu, ada banyak
	pianis yang lebih rendah yang musiknya disambut oleh teman-teman dan
	keluarga mereka.
	Tetapi pianis ini mungkin sedang melakukan sesuatu yang lain untuk
	mendukung diri mereka sendiri, dan demikian pula dengan pemrograman bersamaan.
	
	Satu potensi manfaat membaca secara pasif buku ini adalah kemampuan
	membaca dan memahami kode bersamaan modern.
	Kemampuan ini mungkin pada gilirannya memungkinkan Kamu untuk:

	\begin{enumerate}
	\item 	Melihat apa yang dilakukan kernel sehingga Kamu dapat memeriksa
		apakah kasus penggunaan yang diajukan valid.
	\item 	Mengejar bug kernel.
	\item 	Menggunakan informasi dalam kernel untuk lebih mudah
		mengejar bug userspace.
	\item 	Menghasilkan perbaikan untuk bug kernel.
	\item 	Membuat fitur kernel yang sederhana, baik dari nol atau
		menggunakan metode pengembangan modern copy-pasta.
	\end{enumerate}

	Jika Kamu adalah seorang pemrogram yang berpengalaman, membaca
	buku ini secara pasif mungkin akan memungkinkan Kamu untuk
	menggunakan teknik bersamaan modern dengan lebih baik.
	
	Dan akhirnya, jika pekerjaan Kamu adalah untuk mengkoordinasikan
	kegiatan pengembang yang menggunakan teknik bersamaan modern,
	membaca buku ini secara pasif mungkin akan membantu Kamu
	memahami apa yang mereka bicarakan.
	
}\QuickQuizEnd

Saya belajar ini dengan cara yang keras selama tugas-tugas
untuk gelar saya.
Saya sedang mempelajari topik yang saya kenal, dan saya terkejut
bagaimana sedikitnya dari latihan bab itu yang saya bisa jawab
dari kepala saya.\footnote{
	Jadi saya kira bahwa itu baik-baik saja bahwa dosen saya menolak
	mengizinkan saya untuk menghindari kelas itu!}
Memaksa diri untuk menjawab pertanyaan itu sangat meningkatkan
pemahaman saya.
Jadi dengan kuis cepat ini saya tidak meminta Kamu untuk melakukan
apa yang saya lakukan sendiri.

Akhirnya, penyakit yang paling umum adalah berpikir bahwa
Kamu sudah memahami materi yang ada.

Pertanyaan cepat dapat menjadi obat yang sangat efektif.

\section{Alternatives to This Book}
\label{sec:Alternatives to This Book}
%
\epigraph{Antara dua kejahatan saya selalu memilih satu yang belum pernah saya coba sebelumnya.}
	 {Mae West}

Saat \pplsur{Donald}{Knuth} belajar dengan cara yang keras, Jika Kamu
mau buku Kamu berakhir, Kamu harus fokus.
Buku ini berfokus pada pemrograman pembagian memori, dengan
penekanan pada perangkat lunak yang tinggal di bawah
perangkat lunak stack, seperti kernel sistem operasi,
manajemen data paralel, perpustakaan tingkat rendah, dan sebagainya.
Bahasa pemrograman yang digunakan oleh buku ini adalah C.

Jika kamu lebih tertarik di aspek lain paralelisme, Kamu mungkin
lebih baik disajikan oleh buku lain.
Untungnya, ada banyak alternatif yang tersedia untuk Kamu:

\begin{enumerate}
\item 	Jika kamu lebih suka pembelajaran yang lebih akademis dan ketat
	pemrograman paralel, Kamu mungkin suka buku \pplsur{Maurice P.}{Herlihy}'s
	dan \pplsur{Nir}{Shavit}'s ~\cite{HerlihyShavit2008Textbook,HerlihyShavit2020Textbook}.
	Buku ini dimulai dengan kombinasi yang menarik
	primitif rendah tingkat tinggi abstraksi
	dari perangkat keras, dan bekerja melalui kunci
	dan struktur data sederhana termasuk daftar, antrian,
	tabel hash, dan penghitung, mencapai dengan transaksi
	memory, semua dalam Java.
	\pplsur{Michael}{Scott}'s buku teks~\cite{MichaelScott2013Textbook}
	mengarahkan materi yang sama dengan lebih fokus pada
	perangkat lunak, dan, sejauh yang saya tahu, adalah
	buku teks akademis pertama yang secara resmi diterbitkan dengan
	bagian yang didedikasikan untuk RCU\@.

	\pplsur{Maurice P.}{Herlihy}, \pplsur{Nir}{Shavit},
	\pplsur{Victor}{Luchangco}, dan \pplsur{Michael}{Spear} memang
	mengejar edisi kedua buku teks mereka ~\cite{HerlihyShavit2020Textbook}
	dengan menambahkan bagian-bagian singkat tentang hazard pointers
	dan \acr{rcu}, dengan yang terakhir dalam bentuk \acr{ebr} \@.
	\footnote{
		Meskipun merupakan implementasi yang berisi preemption pembaca
			bug dicatat oleh \pplsur{Richard}{Bornat}.}
	Mereka juga mencakup sejarah singkat keduanya, meskipun dengan
	sejarah singkat \acr{rcu} yang dipotong sekitar setahun setelah
	\acr{rcu} diterima ke dalam kernel Linux dan lebih dari 20 tahun
	setelah karya \pplsur{H. T.}{Kung} dan \pplsur{Philip L.}{Lehman}
	~\cite{Kung80}.
	Para pembaca yang ingin pandangan yang lebih mendalam tentang
	sejarah dapat menemukannya di bab \cref{sec:defer:RCU Related Work}

	Namun, pembaca yang mungkin mencurigai sikap yang tidak
	santun terhadap \acr{rcu} dari penulis buku teks pertama ini
	harus merujuk pada kalimat terakhir pada halaman pertama dari
	satu karyanya~\cite{Balmau:2016:FRM:2935764.2935790}.
	Kalimat itu membaca ``QSBR [sebuah kelas tertentu dari
	implementasi \acr{rcu}] cepat dan dapat diterapkan ke hampir
	semua struktur data.''
	Inilah bukanlah kata-kata seseorang yang tidak suka
	terhadap RCU.
\item	Jika Kamu lebih suka pembelajaran yang lebih akademis 
	pemrograman paralel dari sudut pandang pemrograman\-/bahasa\-/pragmatika,
	Kamu mungkin tertarik pada bab paralelisme dari
	\pplsur{Michael}{Scott}'s
	buku teks~\cite{MichaelScott2006Textbook,MichaelScott2015Textbook}
	pemrograman bahasa pragmatika.
\item	Jika Kamu tertarik pada pola-pola objek-orientasi
	pemrograman paralel yang fokus pada C++,
	Kamu mungkin mencoba Volumes~2 dan~4 dari \pplsur{Douglas C.}{Schmidt}'s POSA
	series~\cite{SchmidtStalRohnertBuschmann2000v2Textbook,
	BuschmannHenneySchmidt2007v4Textbook}.
	Volume~4 khususnya memiliki beberapa bab menarik
	menerapkan pekerjaan ini ke aplikasi gudang.
	Realisme contoh ini dibuktikan oleh
	bagian yang berjudul ``Partitioning the Big Ball of Mud'',
	dalam hal masalah inherent dalam paralelisme seringkali
	mengambil tempat kedua untuk mendapatkan kepala kita
	untuk mengatasi aplikasi dunia nyata.
\item	Jika Kamu ingin bekerja dengan perangkat keras Linux,
	maka \pplsur{Jonathan}{Corbet}'s, \pplsur{Alessandro}{Rubini}'s,
	dan \pplsur{Greg}{Kroah-Hartman}'s
	``Linux Device Drivers''~\cite{CorbetRubiniKroahHartman}
	adalah tidak tergantikan, sebagaimana juga situs web Linux Weekly News
	(\url{https://lwn.net/}).
	Terdapat banyak buku dan sumber daya lainnya
	terkait topik yang lebih umum tentang inti Linux.
\item	Jika Kamu tertarik pada pemrograman paralel yang
	terfokus pada komputasi ilmiah dan teknis,
	maka Kamu mungkin mencoba buku teks \pplsur{Timothy G.}{Mattson} et al.~\cite{Mattson2005Textbook}.
	Buku teks ini membahas Java, C/C++, OpenMP, dan MPI\@.
	Pola-pola yang ditawarkan sangat fokus pada desain
	dan implementasi.
\item	Jika fokus utama kamu adalah komputasi ilmiah dan teknis,
	dan kamu tertarik pada GPU, CUDA, dan MPI,
	maka Kamu mungkin mencoba buku teks \pplsur{Norman L.}{Matloff}~\cite{NormMatloff2017ParProcBook}.
	Begitu juga dengan vendor GPU yang
	memiliki banyak informasi tambahan~\cite{AMD2020ROCm,CyrilZeller2011GPGPUbasics,NVidia2017GPGPU,NVidia2017GPGPU-university}.
\item	Jika Kamu tertarik pada POSIX Threads, Kamu mungkin
	melihat \pplsur{David R.}{Butenhof}'s buku~\cite{Butenhof1997pthreads}.
	Dalam tambahan,
	\pplsur{W.~Richard}{Stevens}'s buku~\cite{WRichardStevens1992,WRichardStevens2013}
	mengulas UNIX dan POSIX, dan \pplsur{Stewart}{Weiss}'s catatan kuliah
	~\cite{StewartWeiss2013UNIX} memberikan
	pengantar yang mendalam dan mudah diakses dengan set contoh yang baik.
\item	Jika Kamu tertarik pada C++11, Kamu mungkin tertarik pada
	\pplsur{Anthony}{Williams}'s ``C++ Concurrency in Action:
	Practical Multithreading''~\cite{AnthonyWilliams2012,AnthonyWilliams2019}.
\item 	Jika Kamu tertarik pada C++, tapi dalam lingkungan Windows,
	maka Kamu mungkin mencoba \ppl{Herb}{Sutter}'s ``Effective Concurrency''
	series di
	Dr.~Dobbs Journal~\cite{HerbSutter2008EffectiveConcurrency}.
	Seri ini melakukan pekerjaan yang wajar dalam mempresentasikan
	pendekatan yang masuk akal terhadap paralelisme.
\item	Jika kamu ingin mencoba Intel Threading Building Blocks,
	maka mungkin buku \ppl{James}{Reinders}'s~\cite{Reinders2007Textbook}
	adalah yang Kamu cari.
\item	Mereka yang tertarik mempelajari bagaimana berbagai jenis multi-prosesor
	perangkat keras
	organisasi cache mempengaruhi implementasi inti
	perlu memeriksa \ppl{Curt}{Schimmel}'s klasik
	pemrosesan topik ini~\cite{Schimmel:1994:USM:175689}.
\item 	Jika kamu mencari tampilan perangkat keras,
	maka \pplsur{John L.}{Hennessy}'s dan \pplsur{David A.}{Patterson}'s
	klasik teks~\cite{Hennessy2017,Hennessy2011} sangat layak dibaca.
	Versi ``Readers Digest'' dari teks ini yang berfokus pada
	pekerjaan ilmiah dan teknis (menghancurkan array besar)
	mungkin ditemukan dalam buku teks \ppl{Andrew}{Chien}'s
	~\cite{AndrewChien2022ComputerArchitectureScientists}.
	Jika kamu mencari buku teks tentang model pengurutan memori
	dari sudut pandang yang lebih hardware-orientasi,
	yang ditulis oleh \ppl{Daniel}{Sorin} et al.~\cite{DanielJSorin2011MemModel,%
	VijayNagarajan2020MemModel}
	dianjurkan.
	Untuk tutorial pengurutan memori dari sudut pandang Linux-kernel,
	\ppl{Paolo}{Bonzini}' Seris LWN adalah tempat yang baik untuk
	mulai~\cite{PaoloBonzini2021lockless1,PaoloBonzini2021lockless2,PaoloBonzini2021lockless3,PaoloBonzini2021lockless4,PaoloBonzini2021lockless5,PaoloBonzini2021lockless6}.
\item 	Akhirnya, mereka yang menggunakan Java mungkin akan
	dilayani dengan baik oleh \ppl{Doug}{Lea}'s
	buku teks~\cite{DougLea1997Textbook,Goetz2007Textbook}.
\end{enumerate}

Namun, jika Kamu tertarik pada prinsip-prinsip desain paralel
untuk perangkat lunak tingkat rendah, terutama perangkat lunak yang ditulis
dalam bahasa C, baca lebih lanjut!

\section{Sample Source Code}
\label{sec:howto:Sample Source Code}
%
\epigraph{Gunakan sumbernya, Luke!}{Penggemar Star Wars tidak dikenal}

Buku ini membahas cukup banyak kode sumber, dan dalam banyak kasus
kode sumber ini dapat ditemukan di direktori \path{CodeSamples}
git tree buku ini.
Misalnya, pada sistem UNIX, Kamu harus dapat mengetikkan:

\begin{VerbatimU}
find CodeSamples -name rcu_rcpls.c -print
\end{VerbatimU}

Perintah ini akan menemukan berkas \path{rcu_rcpls.c}, yang disebutkan dalam
\cref{chp:app:``Toy'' RCU Implementations}\@.
Sistem non-UNIX memiliki cara mereka sendiri untuk menemukan berkas berdasarkan nama.

\section{Whose Book Is This?}
\label{sec:howto:Whose Book Is This?}
%
\epigraph{Jika kamu menjadi seorang guru, oleh muridmu kamu akan diajari}
	 {Oscar Hammerstein II}

Sesuai dengan cover, editor buku ini adalah \ppl{Paul E.}{McKenney}.
Namun, editor menerima kontribusi melalui email list
\href{mailto:perfbook@vger.kernel.org}
{\nolinkurl{perfbook@vger.kernel.org}}.
Kontribusi ini dapat berupa email teks, patch terhadap sumber buku \LaTeX{},
atau bahkan permintaan \co{git pull}.
Gunakan format yang paling cocok untuk Kamu.

Untuk membuat patch atau permintaan \co{git pull}, Kamu akan membutuhkan
sumber \LaTeX{} buku ini, yang terletak di
\url{git://git.kernel.org/pub/scm/linux/kernel/git/paulmck/perfbook.git},
atau, alternatif,
\url{https://git.kernel.org/pub/scm/linux/kernel/git/paulmck/perfbook.git}.
Kamu tentu saja juga akan membutuhkan \co{git} dan \LaTeX{}, yang tersedia
sebagai bagian dari sebagian besar distribusi Linux mainstream.
Paket lain mungkin diperlukan, tergantung pada distribusi yang Kamu gunakan.
Daftar paket yang diperlukan untuk beberapa distribusi populer tercantum
di berkas \path{FAQ-BUILD.txt} di sumber \LaTeX{} buku ini.

\begin{listing}
\begin{VerbatimL}[breaklines=true,breakafter=/,
        breakaftersymbolpre=\raisebox{-.7ex}{\textcolor{darkgray}{\Pisymbol{psy}{191}}},
	breaksymbolleft=\textcolor{darkgray}{\tiny\ensuremath{\hookrightarrow}},
        numbers=none,xleftmargin=0pt]
git clone git://git.kernel.org/pub/scm/linux/kernel/git/paulmck/perfbook.git
cd perfbook
# Kamu mungkin harus menginstall font. Lihat item 1 di FAQ.txt.
make                     # -jN untuk pararel build
evince perfbook.pdf &    # Versi dua kolom
make perfbook-1c.pdf
evince perfbook-1c.pdf & # Versi satu kolom (lebih mudah dibaca e-reader)
make help                # Menampilkan opsi build lainnya
\end{VerbatimL}
\caption{Creating an Up-To-Date PDF}
\label{lst:howto:Creating a Up-To-Date PDF}
\end{listing}

Untuk membuat dan menampilkan pohon sumber \LaTeX{} terkini buku ini,
gunakan daftar perintah Linux yang ditunjukkan di
\cref{lst:howto:Creating a Up-To-Date PDF}.
Dalam beberapa lingkungan, perintah \co{evince} yang menampilkan
\path{perfbook.pdf} mungkin perlu diganti, misalnya dengan \co{acroread}.
Perintah \co{git clone} hanya perlu digunakan pertama kali Kamu membuat PDF,
selanjutnya, Kamu dapat menjalankan perintah yang ditunjukkan di
\cref{lst:howto:Membuat PDF yang diperbarui} untuk menarik perubahan apa pun
dan membuat PDF yang diperbarui.\@
Perintah di \cref{lst:howto:Membuat PDF yang diperbarui} harus dijalankan
dalam direktori \path{perfbook} yang dibuat oleh perintah yang ditunjukkan di
\cref{lst:howto:Creating a Up-To-Date PDF}.

\begin{listing}
\begin{VerbatimL}[numbers=none,xleftmargin=0pt]
git remote update
git checkout origin/master
make                     # -jN Untuk pararel build
evince perfbook.pdf &    # Versi dua kolom
make perfbook-1c.pdf
evince perfbook-1c.pdf & # Versi satu kolom (lebih mudah dibaca e-reader)
\end{VerbatimL}
\caption{Membuat PDF yang diperbarui}
\label{lst:howto:Membuat PDF yang diperbarui}
\end{listing}

PDF dari buku ini diposting secara sporadis di
\url{https://kernel.org/pub/linux/kernel/people/paulmck/perfbook/perfbook.html}
dan di
\url{http://www.rdrop.com/users/paulmck/perfbook/}.

Proses aktual kontribusi patch dan permintaan \co{git pull} sama seperti
proses kontribusi kernel Linux, yang dijelaskan di
\url{https://www.kernel.org/doc/html/latest/process/submitting-patches.html}.
Satu persyaratan penting adalah bahwa setiap patch (atau commit, dalam
kasus permintaan \co{git pull}) harus berisi baris \co{Signed-off-by:} yang
valid, yang memiliki format sebagai berikut:

\begin{VerbatimU}
Signed-off-by: Nama ku <namaku@example.org>
\end{VerbatimU}

Tolong lihat \url{https://lkml.org/lkml/2007/1/15/219} untuk contoh patch
dengan baris \co{Signed-off-by:}.
Perhatikan bahwa baris \co{Signed-off-by:} memiliki makna yang sangat
spesifik, yaitu bahwa Kamu menyatakan bahwa:

\begin{enumerate}[label={(\alph*)}]
\item 	Kontribusi yang dibuat sepenuhnya atau sebagian besar oleh Kamu
	dan Kamu memiliki hak untuk mengirimkannya di bawah lisensi
	sumber terbuka yang tercantum dalam berkas; atau
\item	Kontribusi berdasarkan pekerjaan sebelumnya yang, sejauh yang
	diketahui oleh Kamu, dilindungi oleh lisensi sumber terbuka yang
	sesuai dan Kamu memiliki hak di bawah lisensi tersebut untuk
	mengirimkan pekerjaan tersebut dengan modifikasi, baik yang
	dibuat sepenuhnya atau sebagian besar oleh Kamu, di bawah
	lisensi sumber terbuka yang sama (kecuali jika Kamu diizinkan
	mengirimkan di bawah lisensi yang berbeda), seperti yang
	dicantumkan dalam berkas; atau
\item	Kontribusi disediakan langsung kepadaku oleh orang lain yang
	menyatakan (a), (b) atau (c) dan saya tidak mengubahnya.
\item	Saya memahami dan setuju bahwa proyek ini dan kontribusi
	adalah publik dan bahwa catatan kontribusi (termasuk semua
	informasi pribadi yang saya kirimkan bersamanya, termasuk
	tanda tangan saya) dipertahankan selamanya dan dapat
	didistribusikan konsisten dengan proyek ini atau lisensi
	sumber terbuka yang terlibat.
\end{enumerate}

Ini sangat mirip dengan Sertifikat Pengembang (DCO) versi 1.1 yang
digunakan kernel Linux.
Kamu harus menggunakan nama asli Kamu:
Sayangnya, saya tidak dapat menerima kontribusi yang tidak memiliki nama
asli atau anonim.

Bahasa buku ini adalah bahasa Inggris Amerika, namun, sifat open-source
dari buku ini memungkinkan terjadinya terjemahan, dan saya pribadi
menghargainya.
Lisensi open-source yang melindungi buku ini juga memungkinkan Kamu
untuk menjual terjemahan Kamu, jika Kamu menginginkannya.
Saya meminta Kamu untuk mengirimkan saya salinan terjemahan (hardcopy jika
tersedia), namun, ini adalah permintaan yang saya lakukan sebagai
kebaikan profesional, dan bukan merupakan syarat yang harus dipenuhi
sebelum Kamu memiliki izin yang sudah Kamu miliki di bawah lisensi
Creative Commons dan GPL\@.
Silakan lihat berkas \co{FAQ.txt} di dalam pohon sumber untuk daftar
terjemahan yang sedang berlangsung.
Saya menganggap upaya terjemahan sebagai ``sedang berlangsung'' setelah
setidaknya satu bab telah diterjemahkan sepenuhnya.

Ada banyak gaya di bawah rubrik ``Bahasa Inggris Amerika''.
Gaya untuk buku ini dijelaskan Di
\cref{chp:app:styleguide:Style Guide}.

Seperti dicatat di awal bagian ini, saya editor buku ini.
Namun, jika Kamu memilih untuk berkontribusi, buku ini akan menjadi buku
kamu juga.
Dalam semangat itu, saya menawarkan \cref{chp:Introduction}, bagian
pengantar kami.

\QuickQuizAnswersChp{qqzhowto}

